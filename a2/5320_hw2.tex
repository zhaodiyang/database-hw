%%%%%%%%%%%%%%%%%%%%%%%%%%%%%%%%%%%%%%%%%
% Programming/Coding Assignment
% LaTeX Template
%
% This template has been downloaded from:
% http://www.latextemplates.com
%
% Original author:
% Ted Pavlic (http://www.tedpavlic.com)
%
% Note:
% The \lipsum[#] commands throughout this template generate dummy text
% to fill the template out. These commands should all be removed when 
% writing assignment content.
%
% This template uses a Perl script as an example snippet of code, most other
% languages are also usable. Configure them in the "CODE INCLUSION 
% CONFIGURATION" section.
%
%%%%%%%%%%%%%%%%%%%%%%%%%%%%%%%%%%%%%%%%%

%----------------------------------------------------------------------------------------
%	PACKAGES AND OTHER DOCUMENT CONFIGURATIONS
%----------------------------------------------------------------------------------------

\documentclass{article}

\usepackage{fancyhdr} % Required for custom headers
\usepackage{lastpage} % Required to determine the last page for the footer
\usepackage{extramarks} % Required for headers and footers
\usepackage[usenames,dvipsnames]{color} % Required for custom colors
\usepackage{graphicx} % Required to insert images
\usepackage{listings} % Required for insertion of code
\usepackage{courier} % Required for the courier font
\usepackage{lipsum} % Used for inserting dummy 'Lorem ipsum' text into the template

% Margins
\topmargin=-0.45in
\evensidemargin=0in
\oddsidemargin=0in
\textwidth=6.5in
\textheight=9.0in
\headsep=0.25in

\linespread{1.1} % Line spacing

% Set up the header and footer
\pagestyle{fancy}
\lhead{\hmwkAuthorName} % Top left header
\rhead{\hmwkClass\ %(\hmwkClassInstructor\ \hmwkClassTime):
\hmwkTitle} % Top center head
%\rhead{\firstxmark} % Top right header
\lfoot{\lastxmark} % Bottom left footer
\cfoot{} % Bottom center footer
\rfoot{Page\ \thepage\ of\ \protect\pageref{LastPage}} % Bottom right footer
\renewcommand\headrulewidth{0.4pt} % Size of the header rule
\renewcommand\footrulewidth{0.4pt} % Size of the footer rule

\setlength\parindent{0pt} % Removes all indentation from paragraphs

%----------------------------------------------------------------------------------------
%	CODE INCLUSION CONFIGURATION
%----------------------------------------------------------------------------------------

\definecolor{MyDarkGreen}{rgb}{0.0,0.4,0.0} % This is the color used for comments
\lstloadlanguages{Perl} % Load Perl syntax for listings, for a list of other languages supported see: ftp://ftp.tex.ac.uk/tex-archive/macros/latex/contrib/listings/listings.pdf
\lstset{language=Perl, % Use Perl in this example
        frame=single, % Single frame around code
        basicstyle=\small\ttfamily, % Use small true type font
        keywordstyle=[1]\color{Blue}\bf, % Perl functions bold and blue
        keywordstyle=[2]\color{Purple}, % Perl function arguments purple
        keywordstyle=[3]\color{Blue}\underbar, % Custom functions underlined and blue
        identifierstyle=, % Nothing special about identifiers                                         
        commentstyle=\usefont{T1}{pcr}{m}{sl}\color{MyDarkGreen}\small, % Comments small dark green courier font
        stringstyle=\color{Purple}, % Strings are purple
        showstringspaces=false, % Don't put marks in string spaces
        tabsize=5, % 5 spaces per tab
        %
        % Put standard Perl functions not included in the default language here
        morekeywords={rand},
        %
        % Put Perl function parameters here
        morekeywords=[2]{on, off, interp},
        %
        % Put user defined functions here
        morekeywords=[3]{test},
       	%
        morecomment=[l][\color{Blue}]{...}, % Line continuation (...) like blue comment
        numbers=left, % Line numbers on left
        firstnumber=1, % Line numbers start with line 1
        numberstyle=\tiny\color{Blue}, % Line numbers are blue and small
        stepnumber=5 % Line numbers go in steps of 5
}

% Creates a new command to include a perl script, the first parameter is the filename of the script (without .pl), the second parameter is the caption
\newcommand{\perlscript}[2]{
\begin{itemize}
\item[]\lstinputlisting[caption=#2,label=#1]{#1.pl}
\end{itemize}
}

%----------------------------------------------------------------------------------------
%	DOCUMENT STRUCTURE COMMANDS
%	Skip this unless you know what you're doing
%----------------------------------------------------------------------------------------

% Header and footer for when a page split occurs within a problem environment
\newcommand{\enterProblemHeader}[1]{
\nobreak\extramarks{#1}{#1 continued on next page\ldots}\nobreak
\nobreak\extramarks{#1 (continued)}{#1 continued on next page\ldots}\nobreak
}

% Header and footer for when a page split occurs between problem environments
\newcommand{\exitProblemHeader}[1]{
\nobreak\extramarks{#1 (continued)}{#1 continued on next page\ldots}\nobreak
\nobreak\extramarks{#1}{}\nobreak
}

\setcounter{secnumdepth}{0} % Removes default section numbers
\newcounter{homeworkProblemCounter} % Creates a counter to keep track of the number of problems

\newcommand{\homeworkProblemName}{}
\newenvironment{homeworkProblem}[1][Problem \arabic{homeworkProblemCounter}]{ % Makes a new environment called homeworkProblem which takes 1 argument (custom name) but the default is "Problem #"
\stepcounter{homeworkProblemCounter} % Increase counter for number of problems
\renewcommand{\homeworkProblemName}{#1} % Assign \homeworkProblemName the name of the problem
\section{\homeworkProblemName} % Make a section in the document with the custom problem count
\enterProblemHeader{\homeworkProblemName} % Header and footer within the environment
}{
\exitProblemHeader{\homeworkProblemName} % Header and footer after the environment
}

\newcommand{\problemAnswer}[1]{ % Defines the problem answer command with the content as the only argument
\noindent\framebox[\columnwidth][c]{\begin{minipage}{0.98\columnwidth}#1\end{minipage}} % Makes the box around the problem answer and puts the content inside
}

\newcommand{\homeworkSectionName}{}
\newenvironment{homeworkSection}[1]{ % New environment for sections within homework problems, takes 1 argument - the name of the section
\renewcommand{\homeworkSectionName}{#1} % Assign \homeworkSectionName to the name of the section from the environment argument
\subsection{\homeworkSectionName} % Make a subsection with the custom name of the subsection
\enterProblemHeader{\homeworkProblemName\ [\homeworkSectionName]} % Header and footer within the environment
}{
\enterProblemHeader{\homeworkProblemName} % Header and footer after the environment
}

%----------------------------------------------------------------------------------------
%	NAME AND CLASS SECTION
%----------------------------------------------------------------------------------------

\newcommand{\hmwkTitle}{Homework\ \#2} % Assignment title
\newcommand{\hmwkDueDate}{Wednesday,\ March\ 4,\ 2015} % Due date
\newcommand{\hmwkClass}{CS\ 5320} % Course/class
%\newcommand{\hmwkClassTime}{10:30am} % Class/lecture time
%\newcommand{\hmwkClassInstructor}{Jones} % Teacher/lecturer
\newcommand{\hmwkAuthorName}{Diyang Zhao (dz276), Kai Wang (kw296)} % Your name

%----------------------------------------------------------------------------------------
%	TITLE PAGE
%----------------------------------------------------------------------------------------

\title{
\vspace{2in}
\textmd{\textbf{\hmwkClass:\ \hmwkTitle}}\\
\normalsize\vspace{0.1in}\small{Due\ on\ \hmwkDueDate}\\
\vspace{0.1in}
%\large{\textit{\hmwkClassInstructor\ \hmwkClassTime}}
\vspace{3in}
}

\author{\textbf{\hmwkAuthorName}}
\date{} % Insert date here if you want it to appear below your name

%----------------------------------------------------------------------------------------

\begin{document}

\maketitle

%----------------------------------------------------------------------------------------
%	TABLE OF CONTENTS
%----------------------------------------------------------------------------------------

%\setcounter{tocdepth}{1} % Uncomment this line if you don't want subsections listed in the ToC

%\newpage
%\tableofcontents
\newpage

%----------------------------------------------------------------------------------------
%	PROBLEM 
%----------------------------------------------------------------------------------------

% To have just one problem per page, simply put a \clearpage after each problem

%\begin{homeworkProblem}
%Listing \ref{homework_example} shows a Perl script.

%\perlscript{homework_example}{Sample Perl Script With Highlighting}

%\lipsum[1]

%\problemAnswer{
%\begin{center}
%\includegraphics[width=0.75\columnwidth]{example_figure} % Example image
%\end{center}

%\lipsum[3-5]
%}
%\end{homeworkProblem}

%----------------------------------------------------------------------------------------
%	PROBLEM 
%----------------------------------------------------------------------------------------

\setcounter{homeworkProblemCounter}{1}
\begin{homeworkProblem}

\begin{subsection}{2.1}
(a)\\
Assume: $M$ pages in R, $P_{R}$ tuples per page and $N$ pages in S, $P_{S}$ tuples per page.\\
R is outer relation and S is inner relation.\\
The cost of the Index Nested Loops is $M+M*P_{R}*X$, where X is cost of finding matching S tuples.\\
As we have clustered B+ tree index on S.A, we can assume traverse index to find first leaf page need 2 I/Os and scan and retrieve tuples need worst case $P_{S}$, then the cost X is $2*P_{S}$\\
So, the cost is $M+M*P_{R}*2*P_{S}$\\
\\
(b)\\
Assume: $M$ pages in R, $P_{R}$ tuples per page and $N$ pages in S, $P_{S}$ tuples per page.\\
S is outer relation and R is inner relation.\\
The cost of the Index Nested Loops is $N+N*P_{S}*X$, where X is cost of finding matching R tuples.\\
As we have unclustered B+ tree index on R.A, we would better scan the whole file, then the cost X is $M*P_{R}$\\
So, the cost is $M+M*P_{R}*M*P_{R}$\\
\\
(C)\\
Assume: $M$ pages in R, $P_{R}$ tuples per page and $N$ pages in S, $P_{S}$ tuples per page.\\
R is outer relation and S is inner relation.\\
The cost of the Index Nested Loops is $M+M*P_{R}*X$, where X is cost of finding matching S tuples.\\
As we have hash index on S.A, we can assume hash to find first leaf page need 1 I/O and scan and retrieve tuples need worst case $P_{S}$then the cost X is $1*P_{S}$\\
So, the cost is $M+M*P_{R}*P_{S}$\\
\\
(D)\\
Assume: $M$ pages in R, $P_{R}$ tuples per page and $N$ pages in S, $P_{S}$ tuples per page.\\
We have B buffers in the memory.\\
Then merge-sort S and R needs $2M(log_{B-1}(M/B)+1)+2N(log_{B-1}(N/B)+1)$ and merge join need $M*N*P_{S}*P_{R}$\\
So, the cost is $2M(log_{B-1}(M/B)+1)+2N(log_{B-1}(N/B)+1)+M*N*P_{S}*P_{R}$\\

\end{subsection}

\begin{subsection}{2.2}
The number of passes required in two-way sort $log_{2}(N)+1=16$.\\
Each page is read and written once per pass, then, the cost is $2N(log_{2}(N)+1)=960,000$\\
The 2-way merge sort always uses just 3 buffer pages, it is impossible to sort these files in 2 passes.\\
\end{subsection}

\begin{subsection}{2.3}
(a) Use a (clustered) B+ tree index on attribute R.a.\\
We only need to find a range, so we do not want to scan the whole file. Hash is not good for range, so we use B+ tree.\\
\\
(b) Use a linear hashed index on attribute R.a.\\
We only need to find one pages, so it is easiest way to directly hash it and get the page we want.\\
\\
(c) Use a (clustered) B+ tree index on attribute R.a.\\
\\
(d) Access the sorted file for R directly.\\
We need to scan almost all the pages. Even though we use the tree or hash find the 21600, we still need to scan all of the pages. So it is better to scan pages directly.\\
\end{subsection}

\begin{subsection}{2.4}
(a)\\
(i) As we do not have a clusted B+ plus tree on age, the file scan will be the best way. The cost is 64000.\\
To verify our answer, we can calculate the cost using unclusted B+ plus tree, which is 2 for look up, 64000*0.1*20/160 for select the indexes and 64000*16*0.1 for matching the tuples. This cost is obviously large than file scan.\\
\\
(ii) As we want to find state exactly is ``CA'', we can use the clustered B+ tree index on <state, age>. The cost is 2 for look up, 64000*0.1*20/160 for select the index and 64000*0.1 for matching the tuples. The total cost is 7202.\\
To verify our answer, we can calculate the cost using hash index on state , which is 64000*0.1*16. This cost is obviously large than clusted B+ tree.\\
\\
(iii) Same as (i), the file scan will be the best with the cost of 64000.\\
\\
(iv) Same as (ii), we can use the clustered B+ tree index on <state, age>. The cost is 2 for look up, 64000*0.1*20/160 for select the index and 64000*0.1 for matching the tuples. The total cost is 7202.\\
\\
(b)\\
Same as (a)(ii), we can use the clustered B+ tree index on <state, age>, but we can only scan the index without matching the pages. The cost is 2 for look up, 64000*0.1*20/160 for select the index. The total cost is 802.\\

\end{subsection}

\end{homeworkProblem}   


%----------------------------------------------------------------------------------------

\end{document}